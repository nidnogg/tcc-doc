\begin{resumo}[Abstract]
\begin{otherlanguage*}{english}
\begin{SingleSpace}
Web security has been in constant change and evolution for the past decades, and the constant stream of new initiatives demand a serious level of engagement from the professionals committed to it to adapt to the latest vulnerabilities. In this context, a literature review on the topic of Web Application Hardening has been conducted by the authors in order to foster solutions in the area. Among the research results, a considerable contribution to an open-source security application, \textit{WAF-A-MoLE} has been made, entitled wafamole++. Both have been built as tools for generating adversarial examples for Web Application Firewalls based on Machine Learning against SQL injections, leveraging techniques such as Mutational Fuzzing and contemplating several types of data classifiers. For its elaboration, an open-source firewall of such type was used along with the original data made available by the authors, as well as datasets from platforms such as Kaggle for training new models. In addition, a series of new mutation operators intrinsic to the operation of the base software were created for the generation of richer and more efficient adversarial examples. It is believed that with this expanded release, a wide range of modern Web applications can be enriched with enhancements to their firewalls.
\end{SingleSpace}

%Eventually you can also write it in spanish \textit{(resumen}), french \textit{(résumé)}, italian \textit{(riassunto)} etc.

\vspace{\onelineskip}
   \textbf{Keywords}: web application hardening; security; information security; machine learning; neural networks; firewalls; literature review.
   
 %  latex. abntex. text editoration.
 \end{otherlanguage*}
\end{resumo}




  
