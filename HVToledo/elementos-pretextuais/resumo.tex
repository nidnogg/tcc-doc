\begin{resumo}
\begin{SingleSpace}
Em constante mudança e sempre em relevância para o mercado, a área de segurança Web possui sempre novas iniciativas e exige dos profissionais comprometidos com a mesma um sério nível de engajamento para se adaptar às mais recentes vulnerabilidades. Nesse contexto, buscou-se, por meio de uma revisão de literatura com o tópico de Hardening de Aplicações Web, soluções para serem fomentadas na área pelos autores. Dentre os resultados da pesquisa, gerou-se uma contribuição considerável a uma aplicação open-source de segurança, o WAF-A-MoLE, dando luz a um upgrade denominado wafamole++. Ambos são uma ferramenta de geração de exemplos adversariais para Firewalls de Aplicações Web baseados em Aprendizado de Máquina contra injeções SQL, elencando técnicas como Fuzzing Mutacional e contemplando diversos tipos de classificadores de dados. Para a elaboração da mesma, utilizou-se um firewall \textit{open-source} desse tipo juntamente dos dados originais disponibilizados pelos autores e datasets oriundos de plataformas como o Kaggle para o treinamento de novos modelos. Além disso, uma série de novos operadores de mutação intrínsecos ao funcionamento do software base foram criados para uma geração de exemplos mais ricos e eficientes. Acredita-se que com essa expansão, um leque amplo de aplicações Web modernas possa ser enriquecido com aprimoramentos aos seus firewalls.
%separadas entre si por ponto e finalizadas também por ponto.
\end{SingleSpace}
\vspace{\onelineskip}
\textbf{Palavras-chave}: hardening aplicações web; segurança; infosec; aprendizado de máquina; redes neurais; firewalls; revisão de literatura.
%latex. abntex. editoração de texto.

\end{resumo}

% Palavras-chave separadas e finalizadas por ponto


