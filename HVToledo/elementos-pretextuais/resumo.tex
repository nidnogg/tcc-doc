\begin{resumo}
\begin{SingleSpace}
Em constante mudança e sempre em relevância para o mercado, a área de segurança Web possui sempre novas iniciativas e exige dos profissionais comprometidos com a mesma um sério nível de engajamento para se adaptar às mais recentes vulnerabilidades. Nesse contexto, buscou-se, por meio de uma revisão de literatura com o tópico de \textit{Hardening} de Aplicações Web, soluções para serem complementadas/estendidas na área pelos autores.
Dentre os resultados da pesquisa, gerou-se um complemento a uma aplicação \textit{open-source} de segurança, o \textit{WAF-A-MoLE}, dando luz a um \textit{upgrade} denominado \textit{wafamole++}. O \textit{WAF-A-MoLE} é uma ferramenta de geração de exemplos adversariais para \textit{Firewalls} de Aplicações Web baseados em Aprendizado de Máquina contra injeções SQL, elencando técnicas como Fuzzing Mutacional e contemplando diversos tipos de classificadores de dados, enquanto o \textit{wafamole++} é uma versão estendida do mesmo com um módulo de treinamento, e tratamento de dados, novos modelos de ataque e operadores de mutação. Para a elaboração da ferramenta \textit{wafamole++}, utilizou-se um \textit{firewall }\textit{open-source} desse tipo juntamente dos dados originais disponibilizados pelos autores e \textit{datasets} oriundos de plataformas como o \textit{Kaggle} para o treinamento de novos modelos. Além disso, uma série de novos operadores de mutação intrínsecos ao funcionamento do software base foram criados para uma geração de exemplos mais ricos e eficientes. Acredita-se que com essa expansão, um leque de aplicações Web modernas possa ser enriquecido com aprimoramentos aos seus \textit{firewalls}.
%separadas entre si por ponto e finalizadas também por ponto.
\end{SingleSpace}
\vspace{\onelineskip}
\textbf{Palavras-chave}: hardening aplicações web; segurança; segurança da informação; aprendizado de máquina; redes neurais; firewalls.
%latex. abntex. editoração de texto.

\end{resumo}

% Palavras-chave separadas e finalizadas por ponto


