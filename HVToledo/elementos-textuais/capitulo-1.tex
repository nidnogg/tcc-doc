\chapter{INTRODUÇÃO}

Para o contexto de desenvolvimento de software, segurança da informação é a área que potencialmente define entre o sucesso ou fracasso de uma aplicação. E em aplicações web com elevado tráfego, a preocupação com a segurança pode ter uma manifestação tardia e comprometer o uso do ambiente. Atualmente navegadores web como \textit{Chrome}, \textit{Firefox} e \textit{Safari} ativamente notificam o usuário de falhas graves nesse quesito como certificação HTTPS, tornando ações fundamentais como transações (quando inseguras) na web uma área de atenção redobrada para consumidores finais. Nela, ações de segurança são um recurso crítico e sensível que faz com que poucos sites discorram sobre os seus segredos de segurança, até como uma estratégia contra \textit{hackers}. Isso torna complicado assegurar como é o estado atual de vulnerabilidades web hoje em dia. Nesse sentido, este trabalho procura elencar o estado atual das medidas conhecidas para conter o avanço de ataques às aplicações web e introduzir uma nova ferramenta de código aberto como contribuição adicional para fortalecer equipes de segurança contra as vulnerabilidades associadas aos sítios web. 

Historicamente, o conceito de vulnerabilidade na Internet modificou-se consideravelmente. A \textit{World Wide Web} consistia, nos seus primórdios, de web sites cujas funções principais eram essencialmente fornecer ao usuário final documentos estáticos de informação, surgida inicialmente nos anos 90. O fluxo de informação era de via única, no qual o conteúdo saía do servidor ao usuário final, apenas. Autenticação não era uma realidade para a maioria dos sites, pois não havia necessidade. Cada usuário que visitasse uma página web recebia a mesma informação. \cite{stuttard_web_nodate} Vulnerabilidades nessa época consistiam principalmente em numerosos \textit{exploits} \footnote{Em português significa explorar, significando usar algo para própria vantagem. É um software ou sequência de comandos que toma vantagem de um defeito em uma aplicação para causar um comportamento imprevisto tipicamente com a finalidade de obter controle de tal aplicação. } em cima de software de servidores, para distribuir software de computador ilegalmente copiado com seus códigos de proteção desativados \footnote{A desativação dos códigos de proteção é importante para que a cópia possa ser distribuída sem dificuldades.} (do inglês: \textit{warez}) ou meramente deturpar a parte visual de um site (prática conhecida como \textit{defacement}).

Na atualidade a maioria dos sites na Internet são aplicações robustas com diversas funcionalidades que eram inconcebíveis na época, dependendo de um fluxo de informação de via dupla entre cliente e servidor. Usuários precisam de credenciais para realizar cadastros que cuidam das mais modestas trocas de informações como postagens em fóruns de discussão até os mais sensíveis dados bancários transacionais. Surge a necessidade concreta de gerenciadores de senha de ponta, muitas vezes oferecidos pelo próprio navegador web (embora inseguro). Além disso, torna-se onipresente a linguagem de script/programação \textit{JavaScript} moderna, que fornece a essas aplicações um dinamismo inerente, tanto na parte visual do cliente (\textit{front-end}), como na parte funcional do servidor (\textit{back-end}, através de tecnologias como Node.js). Cada internauta torna-se diferente do próximo, tornando a interação com o site totalmente dependente do usuário, o que afeta o conteúdo que é acessado a cada instante.

O uso maciço da internet pela população traz inúmeras possibilidades de ataques maliciosos, com exploração constante de novas técnicas aproveitando as vulnerabilidades inerentes ao ambiente compartilhado da internet. Um tipo de componente amplamente usado para mitigar vulnerabilidades em uma dada aplicação web é um \textit{Firewall} de Aplicação Web (em inglês: \textit{Web Application Firewall}, comumente abreviado como WAF) - responsável por monitorar tráfego de internet que é conduzido pela aplicação, permitindo ou bloqueando pacotes que possam ser potencialmente maliciosos. E recentemente, com o advento de técnicas de Aprendizado de Máquina, uma série de \textit{firewalls} de Aplicações Web baseados nesse subcampo de Inteligência Artificial foram criados com o intuito não apenas de explorar a área em si, como também de gerar \textit{Firewalls} mais robustos e capazes de deter tráfego de ataques web. 

\textit{Firewalls} propriamente ditos são softwares ou até mesmo hardwares que são responsáveis por examinar e filtrar a informação que atravessa uma determinada conexão. WAFs monitoram dados que uma aplicação web recebe, permitindo averiguar suas vulnerabilidades. Há várias maneiras de testar WAFs hoje quanto a eficácia, tanto manualmente através de documentos padronizados como o \textit{wafec} \cite{wafec_doc}, como através de ferramentas como o \textit{GoTestWAF} \cite{gotestwaf_wallarm}. Mas nem toda ferramenta de teste é capaz de avaliar e/ou realçar a eficácia dos WAFs baseados em Aprendizado de Máquina para a filtragem de informações.

Uma das ferramentas com a capacidade de testar WAFs baseados em Aprendizado de Máquina é o \textit{WAF-A-MoLE} \cite{valenza_waf--mole_2020} que permite o uso de técnicas (detalhadas no Capítulo 2 adiante) de \textit{fuzzing} \cite{fuzzing_book} para explorar as fraquezas de um WAF e com isso apontar caminhos para tornar os ambientes mais robustos.

Essa trabalho procurou complementar a ferramenta \textit{WAF-A-MoLE} em uma série de falhas verificadas pelos autores - uma arquitetura engessada que dificultava o uso e reprodução da ferramenta, uma série de modelos de ataque faltantes que poderiam ser utilizados porém não foram, e uma lista de operadores de mutação (um aspecto chave do seu funcionamento, explicado mais adiante) que poderia ser ainda mais estendida. 

Para mitigar essas dificuldades de desenvolvimento e uso, foram criadas várias ferramentas auxiliares que possibilitam a um contribuidor ou usuário treinar (e testar) um WAF baseado em aprendizado de máquina mais facilmente, uma interface para qualquer classificador da biblioteca \verb+scikit-learn+ \footnote{scikit-learn é uma biblioteca para a linguagem de programação \textit{Python} especializada em análise de dados, com uma suíte de ferramentas de Aprendizado de Máquina. Contém vários algoritmos essenciais como os classificadores, alguns dos quais serão explorados no Capítulo 2.} foi introduzida, e a eficiência do programa foi incrementada com novos operadores de mutação.

Resumidamente, esse complemento é intitulado de \textit{wafamole++}. É uma ferramenta em linha de comando que, com o recebimento de um modelo de um WAF baseado em Aprendizado de Máquina pelo usuário, gera o que é conhecido como exemplo adversarial. Exemplos adversariais para o \textit{wafamole++} são entradas especializadas criadas com o intuito de confundir um WAF. Em Aprendizado de Máquina, eles possuem um significado mais amplo por confundir o componente conhecido como classificador, mais detalhado no Capítulo 2.

Nesse caso, a cada execução dessa ferramenta, gera-se uma entrada que o WAF considera como inócua, porém é uma entrada maliciosa que pode comprometer um sistema que esse WAF estaria a defender. A ferramenta é montada de uma maneira que o usuário possa adaptá-la para avaliar seu próprio WAF contra vulnerabilidades e então treiná-lo novamente com entradas maliciosas para que ele seja capaz de prevê-las melhor.

E nesse contexto, o trabalho está organizado em seis capítulos: uma revisão de literatura sobre a área e trabalhos relacionados é amostrada, traçando comparações com a ferramenta que foi escolhida como alvo de extensão/implementação do trabalho. Posteriormente, é feito um detalhamento sobre a implementação, arquitetura e conceitos por trás da extensão do \textit{WAF-A-MoLE}, intitulada de \textit{wafamole++}. Após isso, um compilado de experimentos responsáveis pela avaliação do \textit{wafamole++}, e seus resultados. Finalmente, as conclusões acerca dos resultados, dificuldades encontradas e apêndices.

\bigskip
