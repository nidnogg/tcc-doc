\chapter{PROJETO DE TESTES REALIZADO}
\label{chp:capitulo5}

\section{Abordagem de testes}
O propósito inicial dos testes realizados é compreender amplamente o impacto das adições ao programa. Para tal, uma série de métricas foram empregadas para avaliação. A primeira característica avaliada foi a execução do programa, de uma maneira exaustiva, para todos os modelos relevantes (excluindo-se apenas o SQLiGOT, por motivos já mencionados. Através dela foi possível obter uma ideia do impacto que os novos modelos têm, e como o wafamole++ se adapta com \textit{wrappers} novos.

O parâmetro principal para tal avaliação foram \textbf{500} execuções por modelo no wafamole++. Tanto os modelos antigos como os modelos novos foram examinados. As medidas bases extraídas foram:
\begin{alineas}
\item \textbf{Número de rodadas restantes} - A partir dessas obtém-se o número de rodadas ou iterações consumidas;
\item \textbf{Menor probabilidade de payload encontrada} - É possível medir com essas a precisão de um determinado modelo;
\item \textbf{Tempo de execução em segundos} - Tanto para o tempo médio de execução como para diagnosticar a robustez. Um tempo maior indica uma maior eficiência em virtude da reforço do dataset;
\item \textbf{Execuções com iterações máximas alcançadas} - Quanto mais ocorrências dessa medida, mais robusto é o modelo uma vez que não é gerado um exemplo adversarial na execução.
\end{alineas}

