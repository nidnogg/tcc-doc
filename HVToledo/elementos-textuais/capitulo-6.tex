\chapter{CONSIDERAÇÕES FINAIS}
\label{chp:capitulo6}
O valor acadêmico e profissional de aplicações como as descritas nesse trabalho foi extensivamente discutido, elucidando possíveis direções da área de Segurança da Informação. Foram analisadas diversas propostas e repositórios com ferramentas e soluções de \textit{hardening} no geral, com muitas sendo descartadas para contribuição ou como significativas. 

Nesse contexto foi enfim encontrada uma potencialmente significativa contribuição para o progresso do campo, culminando no desenvolvimento do wafamole++. O mesmo traz não só modelos e operadores de mutação novos em comparação a sua edição original (WAF-A-MoLE), mas uma adaptação para um WAF open-source (\verb+MLBasedWAF+) baseado em Aprendizado de Máquina que possui uma performance comparável ao \verb+SQLiGOT+ (um modelo de ponta de alta complexidade e robustez). A diferença chave entre ambos, no entanto, é uma implementação consideravelmente mais simples para desenvolvedores e de código aberto ao público.

Do projeto de testes com milhares de execuções realizadas, pode-se determinar que o reforço do dataset \verb+SQLiv3.json+ com exemplos adversariais do wafamole++ nos modelos \verb+test_svc_classifier_moled.dump+ e \verb+test_svc_classifier_extra_moled.dump+ possui um impacto direto na integridade do modelo, progressivamente tornando mais difícil de penetrar o resultado final conforme mais exemplos são adicionados.

Conclui-se nesse âmbito que os modelos \verb+test_ada.dump+ e \verb+test_svc_classifier_extra_moled.dump+ se mostram como prova real de que classificadores em Firewalls desse subcampo podem ser suficientemente reforçados com o wafamole++.

Com o lançamento dessa aplicação, também são bem-vindas contribuições de desenvolvedores interessados, mantendo-se as mesmas diretrizes de expansão do WAF-A-MoLE original. Espera-se que para especialistas na área possam enriquecer com essa ferramentas seus Web Application Firewalls com exemplos geridos pelo mesmo.

\section{Obstáculos Encontrados}

Na etapa de pesquisa, muitas vezes algumas revistas de artigos se mostraram inacessíveis apenas com o vínculo da Universidade via Scopus (complementado pelo CAFe) sendo necessário uma troca exaustiva de plataformas. Sem atrasos desse tipo na pesquisa, é possível que a fase de desenvolvimento do trabalho possa ter sido adiantada e enriquecida com mais casos de teste.

Certamente um obstáculo de destaque que marcou o desenvolvimento da solução wafamole++ foi a série de limitações impostas aos usuários do plano gratuito na plataforma Google Colab. Uma vez sendo necessário em muitos casos um tempo de execução maior que o limite pré-definido de 12 horas da mesma, fica a reflexão acerca de como seriam os resultados se fosse possível treinar os modelos que exigiam tal poder de processamento com os dados originais do WAF-A-MoLE.

Fora essa restrição no plano da plataforma, o trabalho em torno de dados mostrou-se uma barreira para a relevância da implementação. Não só foi necessário uma série de funções auxiliares para realizar Após o fornecimento dos dados originais do WAF-A-MoLE pelos autores, os resultados anteriormente acumulados com o dataset amostrado do Kaggle se tornaram mais valiosos por terem mais uma comparação em suas métricas.

\section{Futuros desenvolvimentos}
Acredita-se a natureza extensível tanto do WAF-A-MoLE como do wafamole++ permite a ambos um potencial de melhoria razoavelmente amplo. Dado que os modelos novos são inteiramente amostrados da biblioteca \verb+scikit-learn+, cabe a reflexão pelos autores de como seriam resultados oriundos de modelos criados em bibliotecas como \verb+Keras+ (em uma implementação mais elaborada que a vista no WAF-Brain), \verb+PyTorch+, e afins. Não só isso, como há também uma infinidade de operadores de mutação a serem explorados, com os três escolhidos no wafamole++ sendo apenas uma breve amostragem de pesquisa dos autores.

Como também é demonstrado nos iPython Notebooks anexados, a otimização de hiperparâmetros via \verb+GridSearch+ poderia ser mais explorada em um ambiente sem restrições de hardware e tempo de uso como o Google Colab, nos modelos com dados originais do WAF-A-MoLE que eram mais exigentes. 
