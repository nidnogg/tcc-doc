\chapter{CONCLUSÃO}
\label{chp:capitulo6}

\section{Obstáculos Encontrados}

Certamente um obstáculo de destaque que marcou o desenvolvimento da solução \textit{wafamole++} foi a série de limitações impostas aos usuários do plano gratuito na plataforma Google Colaboratory. Uma vez sendo necessário em muitos casos um tempo de execução maior que o limite pré-definido de 12 horas da mesma, fica a reflexão acerca de como seriam os resultados se fosse possível treinar os modelos que exigiam tal poder de processamento com os dados originais do \textit{WAF-A-MoLE}.

Além dessa restrição no plano de uso gratuito da plataforma, o trabalho em torno de dados também trouxe uma série de demandas inesperadas. Foi necessário uma série de funções auxiliares para tratar os dados originados do Kaggle, que vieram inconsistentes. Após o fornecimento dos dados originais do \textit{WAF-A-MoLE} pelos autores, os resultados anteriormente acumulados com o conjunto de dados do Kaggle se tornaram mais valiosos por darem uma perspectiva nova de comparação na etapa de testes. Também foram relevantes pois tornaram possível treinar os modelos de SVM não linear \verb+svc_mole+ e \verb+svc_extra_moled+, uma vez que a execução do treinamento com os dados originais excedeu o tempo limite do Google Colaboratory e recursos locais.

\section{Resultados}

Com esse trabalho foi enfim encontrada uma contribuição para o progresso do campo, culminando no desenvolvimento do \textit{wafamole++}. O mesmo traz não só modelos e operadores de mutação novos em comparação a sua edição original (\textit{WAF-A-MoLE}), mas uma adaptação para um WAF de código aberto (\verb+MLBasedWAF+) baseado em Aprendizado de Máquina que possui uma performance comparável ao \verb+SQLiGOT+ (um modelo de ponta, de alta complexidade e robustez). A diferença chave entre ambos, no entanto, é uma implementação mais simples para desenvolvedores e de código aberto ao público.

Nesse contexto, tendo em mente o projeto de testes com milhares de execuções realizadas, pode-se determinar que o reforço do conjunto de dados \verb+SQLiv3.json+ com exemplos adversariais do \textit{wafamole++} nos modelos \verb+test_svc_classifier_moled.dump+ e \linebreak \verb+test_svc_classifier_extra_moled.dump+ possui um impacto direto na integridade do modelo e do WAF avaliado, progressivamente tornando mais difícil de achar vulnerabilidades neles conforme mais exemplos são adicionados.

Conclui-se nesse âmbito que os modelos baseados no AdaBoost e o SVC reforçado (\textit{extra moled}) se mostram como prova real de que classificadores em \textit{firewalls} desse subcampo podem ser suficientemente reforçados com o \textit{wafamole++}.


\section{Trabalhos Futuros}
Acredita-se a natureza extensível tanto do \textit{WAF-A-MoLE} como do \textit{wafamole++} permite a ambos um potencial de melhoria razoavelmente amplo. Dado que os modelos novos são inteiramente amostrados da biblioteca \verb+scikit-learn+, cabe a reflexão pelos autores de como seriam resultados oriundos de modelos criados em bibliotecas como \verb+Keras+ (em uma implementação mais elaborada que a vista no \textit{WAF-Brain}), \verb+PyTorch+, e afins. Não só isso, como há também uma infinidade de operadores de mutação a serem explorados, com os três escolhidos no \textit{wafamole++} sendo apenas uma simples escolha por parte dos autores.

Como também é demonstrado nos iPython Notebooks do repositório final localizados na pasta \verb+models/custom/svc+, a otimização de hiperparâmetros via \verb+GridSearch+ poderia ser mais explorada em um ambiente sem restrições de hardware e tempo de uso como o Google Colab, nos modelos com dados originais do \textit{WAF-A-MoLE} que eram mais exigentes. 

\section{Considerações Finais}
O valor acadêmico e profissional de aplicações como as descritas nesse trabalho foi extensivamente discutido, elucidando possíveis direções da área de Segurança da Informação. Foram analisadas diversas propostas e repositórios com ferramentas e soluções de \textit{infosec} no geral, com muitas sendo descartadas para contribuição ou como significativas. 

Com o lançamento dessa aplicação, também são bem-vindas contribuições de desenvolvedores interessados, mantendo-se as mesmas diretrizes de expansão do \textit{WAF-A-MoLE} original. Espera-se que especialistas na área possam enriquecer com essa ferramenta seus WAFs, fazendo uso dos exemplos que são gerados por ela.
