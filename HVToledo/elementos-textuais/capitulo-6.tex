\chapter{CONSIDERAÇÕES FINAIS}
\label{chp:capitulo6}


Onde se expõe o fechamento das ideias do estudo, são apresentados os resultados da pesquisa, e partindo da análise destes resultados, tiram-se as conclusões e se for necessário, as sugestões relativas ao estudo. \\

Observação: É opcional a apresentação dos desdobramentos relativos à importância, síntese, projeção, repercussão, encaminhamento e outros.

% REFERENCES
Alíneas e subalíneas.
\bigskip

\begin{alineas}
\item linha 1:
\begin{alineas}
\item subalinea 1;
\item subalinea 2;
\end{alineas}
\item linha 2:
\begin{subalineas}
\item subalinea 1;
\item subalinea 2;
\end{subalineas}
\item linha 3:
\begin{incisos}
\item subalinea 1;
\item subalinea 2;
\end{incisos}
\item linha 4.
\end{alineas}
